\documentclass{article}


\usepackage[utf8]{inputenc}
\usepackage[english]{babel}
\usepackage{fancyhdr}
\usepackage{lastpage}
\usepackage{hyperref}
\usepackage{titling}
\usepackage{titlesec}
\usepackage{graphicx}
\usepackage{wrapfig}
\usepackage{listings}
\usepackage{textcomp}
\usepackage{scrextend}
\usepackage{kotex}
%\graphicspath{ {/} }

\pagestyle{fancy}
\fancyhf{}

\lstset{basicstyle=\ttfamily,
escapeinside={||},
mathescape=true}

% last page seems to need two runs, which I assume actually means you need
% to create the bulid and render files in a certain order
% https://tex.stackexchange.com/questions/28708/why-does-pagereflastpage
%     -give-me-rather-than-page-number-of-the-last-pag
\rfoot{\thepage \hspace{1pt} of \pageref{LastPage}}

% define subtitle command
\newcommand{\subtitle}[1]{%
    \posttitle{%
        \par\end{center}
        \begin{center}\large#1\end{center}
        \vskip0.5em}%
}

% define paragraph to be a subsubsubsection
\titleclass{\subsubsubsection}{straight}[\subsection]

\newcounter{subsubsubsection}[subsubsection]
\renewcommand\thesubsubsubsection{\thesubsubsection.
    \arabic{subsubsubsection}}
\renewcommand\theparagraph{\thesubsubsubsection.\arabic{paragraph}}
    % optional; useful if paragraphs are to be numbered

\titleformat{\subsubsubsection}
  {\normalfont\normalsize\bfseries}{\thesubsubsubsection}{1em}{}
\titlespacing*{\subsubsubsection}
{0pt}{3.25ex plus 1ex minus .2ex}{1.5ex plus .2ex}

\makeatletter
\renewcommand\paragraph{\@startsection{paragraph}{5}{\z@}%
  {3.25ex \@plus1ex \@minus.2ex}%
  {-1em}%
  {\normalfont\normalsize\bfseries}}
\renewcommand\subparagraph{\@startsection{subparagraph}{6}{\parindent}%
  {3.25ex \@plus1ex \@minus .2ex}%
  {-1em}%
  {\normalfont\normalsize\bfseries}}
\def\toclevel@subsubsubsection{4}
\def\toclevel@paragraph{5}
\def\toclevel@paragraph{6}
\def\l@subsubsubsection{\@dottedtocline{4}{7em}{4em}}
\def\l@paragraph{\@dottedtocline{5}{10em}{5em}}
\def\l@subparagraph{\@dottedtocline{6}{14em}{6em}}
\makeatother

\setcounter{secnumdepth}{4}
\setcounter{tocdepth}{4}


\title{Steem}
\subtitle{인센티브 및 블록체인 기반의 소셜 미디어 플랫폼}
\date{2017년 08월}
\author{번역: Renohq \\ 감수: 조재우 (clayop) \\ 17.08 개정 번역: 이태민 (tmkor)}

\pagenumbering{arabic}

\begin{document}

    \renewcommand \thesection{\roman{section}}

    \maketitle

    \newpage

    \section{초록}

Steem은 커뮤니티 생성 및 암호화폐 거래를 지원하는 블록체인 데이터베이스이다. Steem은 다양한 암호화폐들과 관련 커뮤니티들의 생성 과정에서 얻은 교훈을 바탕으로 소셜 미디어의 기본 개념을 적용하고 있다. 커뮤니티, 화폐 혹은 자유시장 경제의 참여를 독려하기 위한 핵심 요소는 각 커뮤니티 구성원의 기여도를 일관되게 반영할 수 있는 공정한 회계 제도이다. Steem은 수많은 커뮤니티 구성원들의 개인 기여도를 정확하고 투명하게 보상하고자 하는 최초의 암호화폐다.
    \newpage

    \section{Table of Contents}

    \tableofcontents

    \newpage

    \setcounter{section}{0}

    \renewcommand \thesection{\arabic{section}}

    \section{서론}

        \paragraph{}
            사용자가 생성한 콘텐츠는 Reddit, Facebook, Twitter 등 소셜 미디어 기업들의 주주들에게 총 수십억 달러에 달하는 가치를 창조해 왔다. \textbf{2014년, Reddit은 글 등록, 댓글 쓰기, 투표 등을 통해 reddit.com에 기여하는 모든 사람들에게 그들의 기여에 대해 Reddit, Inc 주식으로 보상할 경우 자사 플랫폼이 더욱 개선될 것이라는 가설을 제기하였다\footnote{레딧의 가상화폐, 포브스, 에리카 머피, 2014년 10월, 
\newline\url{http://www.forbes.com/sites/erikamorphy/2014/10/01/reddits-cryptocurrency-could-have-many-uses/\#4e07b05332b9}}.} Steem은 소셜 미디어 및 온라인 커뮤니티 구성원들 중에서 기여도가 높은 사람들을 암호화폐로 보상함으로써 해당 커뮤니티들을 지원하는 것을 목표로 한다. 이 과정에서 아직 암호화폐 경제에 참여하지 않고 있는 사람들에게까지 유통될 수 있는 화폐가 생성된다.
            
        \paragraph{}
            Steem의 구조를 설명하기 위하여 핵심 원칙들이 적용되었다. 가장 중요한 원칙은 벤처 기업의 성장에 기여한 모든 사람들이 기여도에 따라 그에 상응하는 지분, 현금 혹은 차입금을 해당 벤처 기업으로부터 받아야 한다는 것이다. 이 원칙은 모든 벤처 기업들의 설립 및 유상 증자 시 적용되는 주식 배분 원칙과 동일하다.

        \paragraph{}
            두 번째 원칙은 방식과 관계없이 자본은 동일한 가치를 갖는다는 것이다. 즉, 현금을 출자한 사람들에 못지않게 콘텐츠 제작 및 큐레이션에 시간과 노력을 투자한 사람들도 중요하다. 이를 노동 지분(Sweat Equity) 원칙\footnote{노동 지분(Sweat Equity), Investopedia,
\newline\url{http://www.investopedia.com/terms/s/sweatequity.asp}}이라 부르며, 이전 암호화폐들이 성공하지 못했던 주요 원인이기도 하다.

        \paragraph{}
            세 번째 원칙은 커뮤니티 자체적으로 구성원들을 위해 특정 상품을 생산한다는 것이다. 이 원칙은 커뮤니티 구성원들에게만 특정 상품 혹은 서비스를 판매하는 신용조합, 식품 협동조합, 의료비 공동 부담 제도 등에 적용되고 있다.

        \paragraph{}
            Steem 커뮤니티는 구성원들에게 아래와 같은 서비스들을 제공한다

        \begin{enumerate}
            \item 다양한 맞춤형 뉴스 및 해설
            \item 개인적인 질문들에 대한 질 높은 답변 제공
            \item 미국 달러 환율에 안정적으로 고정된 암호화폐
            \item 지급 수수료 면제
            \item 다른 구성원들에게 위 서비스들을 제공하는 업무와 관련된 취업 기회
        \end{enumerate}

        \paragraph{}
            Steem이 지향하는 경제적 인센티브의 재조정은 이전의 소셜 미디어 및 암호화폐 플랫폼들보다 좀 더 공정하고 포괄적인 결과물을 산출할 것으로 기대된다. 본 백서는 기존의 경제적 인센티브 제도들을 분석하고, Steem 인센티브 제도의 장점을 설명할 것이다.

        \subsection{기여도 평가 방법}

            \paragraph{}
                Steem은 소셜 미디어 기반 경제가 받아들여지고수익을 창출할 수 있도록 하는 것을 가로 막는 주요 장벽들을 해결하는 데 핵심 역량을 집중하고 있다. 당사의 가설은 주요 소셜 미디어 플랫폼들의 성장을 견인한 주요 기법들을 암호화폐의 성공에도 동일하게 적용할 수 있다는 것이다. 암호화폐 기반의 경제적 인센티브 제도는 신규 소셜 미디어 플랫폼의 성장을 활성화할 수 있다. Steem은 암호화폐와 소셜 미디어 간의 시너지 효과를 통해 시장을 선도할 것으로 기대된다.


            \paragraph{}
                Steem이 직면하고 있는 도전과제는 커뮤니티 구성원들에게 공정성을 인정받을 수 있는 개인 기여도 평가 알고리즘을 개발하는 것이다. 이상적으로는, 커뮤니티 구성원들이 서로의 기여도를 평가하여 그에 상응하는 보상을 결정할 수도 있을 것이다. 현실적으로는, 사익을 위한 의도적인 조작을 방지할 수 있는 알고리즘이 개발되어야 한다. 평가 체계의 악용으로 인해 커뮤니티 구성원들이 경제 체계의 공정성을 의심할 수 있다.


            \paragraph{}
                기존 플랫폼들은 사용자 1인당 1표 원칙을 기준으로 운영된다. 이러한 원칙 때문에 Sybil 공격을 통해 평가를 조작하고 서비스 제공자들이 그러한 조작에 가담한 사람들을 파악하여 차단해야 하는 상황이 발생한다. 이미 웹 트래픽 혹은 검열을 기준으로 보상이 이뤄지는 Reddit, Facebook 및 Twitter의 평가 알고리즘들을 조작하려는 시도가 있었다.


            \paragraph{}
                Steem 플랫폼의 기본 거래 단위는 암호화폐 토큰인 STEEM이다. Steem은 1 STEEM당 1표 원칙을 기준으로 운영된다. 이 모델에서는, 계정 잔고에 따라 플랫폼에 대한 기여도가 가장 높은 사람들이 기여도 평가에 가장 큰 영향을 미친다. 게다가, 장기 보유에 대하여 STEEM으로만 투표가 가능하도록 설계되어 있다. 이 모델에서는, 구성원들이 보유하고 있는 STEEM의 장기적인 가치를 극대화할 수 있도록 투표하게 만드는 금전적인 인센티브가 있다.

            \paragraph{}
                Steem 의 기본 개념은 비교적 단순하다: \textit{모든 구성원들을 커뮤니티 기여도에 따라 평가한다는 개념이다.} 본인 기여도에 따라 평가가 공정하게 이뤄질 경우, 구성원들의 기여가 지속되며 해당 커뮤니티의 성장으로 이어질 것이다. 커뮤니티 내부의 불평등은 지속될 수 없다. 결국, 기여도가 낮은 사람들을 지원하던 우수 구성원들이 염증을 느끼며 해당 커뮤니티를 떠나게 된다.

            \paragraph{}
                위에 언급된 도전과제를 극복하려면 수많은 구성원들에게서 요구되는 기여도와 그러한 기여도를 상대적으로 평가할 수 있는 체계를 구축해야 한다.

            \paragraph{}
                검증된 기여도 평가/보상 체계는 자유시장이다. 자유시장은 모든 사람들이 서로 거래를 하고 거래 이익/손실에 따라 보상이 분배되는 단일 커뮤니티로 간주될 수 있다. 자유시장 체계에서는 다른 사람들에게 가치를 제공한 사람들이 보상을 받고 그렇지 못한 사람들은 불이익을 입는다. 자유시장은 다양한 화폐들을 인정하며 현금은 편리한 교환 수단일 뿐이다.

            \paragraph{}
                자유시장은 이미 검증된 체계이므로, 콘텐츠 소비자들이 직접 콘텐츠 제공자들에게 대가를 지불하는 자유시장 체계를 구축하는 것이 매력적으로 보일 수 있다. 하지만, 직접적인 대가 지불은 비효율적이며, 콘텐츠 생성 및 큐레이션에 현실적으로 적용이 불가능하다. 대부분의 콘텐츠의 가치는 후원 의사가 거의 없는 독자들이 인식하는 금전적인 기회 비용에 비해 너무 낮다. 무료 대안들의 풍부함이란 '지불 장벽'으로 인해 독자들이 다른 대안들을 찾게 됨을 의미한다. 기사 당 소액 결제 시스템을 구축하려는 수많은 시도가 있었으나 성공한 사례는 거의 없다.

            \paragraph{}
                Steem은 정량화를 기반으로 모든 유형의 기여도에 대하여 효과적인 소액 결제가 가능하도록 설계되어 있다. 독자들은 더 이상 지불 의사를 결정할 필요가 없으며, 그 대신에 콘텐츠의 “좋아요/싫어요”를 투표함으로써 그에 상응하는 보상을 받을 수 있다. 이는 기존의 소액 결제 및 후원 플랫폼에서 인지된 금전적인 기회 비용 없이 사용자들에게 친숙한 인터페이스를 제공함을 의미한다.

            \paragraph{}
                커뮤니티 구성원들의 투표 데이터는 Steem이 보상금을 정확히 배분하는 데 있어서 매우 중요한 역할을 하다. 따라서, 투표는 유의미한 기여로 인정되어 그에 상응하는 보상이 주어진다. Slashdot과 같은 일부 플랫폼들의 경우, 메타 중재(meta-moderation)\footnote{\textbf{메타 중재}란 2차적인 중재를 말한다. 사용자들은 중재 효율성을 개선하기 위하여 중재자의 의사 결정을 평가한다.
\newline\url{https://en.wikipedia.org/wiki/Meta-moderation_system}} 기법을 활용하여 정직한 중재자들을 평가하고 보상한다. Steem의 경우, 특정 콘텐츠의 홍보에 대한 기여도가 가장 높은 사람들을 보상할 뿐 아니라 콘텐츠 제공자에게 지급된 보상금에 따라 해당 콘텐츠에 투표한 투표자들도 함께 보상한다.
    
            \paragraph{}
                Steem이 객관적으로 인정하고 보상하는 기여도는 다음과 같이 그 종류가 매우 다양하다: 거래 검증, 작업 증명 마이닝, 유동성 보상, 결함이 있는 블록 생성자들의 보고 등.

    \section{기여 방법}

        본 단락에서는 Steem의 기본 구조와 Steem 커뮤니티의 기여도 보상 체계를 다루게 된다.

        \subsection{자본 출자}

            \paragraph{}
                커뮤니티는 두 가지 방법으로 자금을 조달한다: 타인자본 조달 및 자기자본 조달. 자기자본 출자자들은 커뮤니티가 성장하면 이익을 얻지만 쇠퇴하면 손실을 보게 된다. 타인자본 출자자들은 일정 금액의 이자를 보장받지만, 커뮤니티의 성장과 관련된 이익에는 아무런 권리도 갖지 못한다. 커뮤니티 성장 및 화폐 가치 상승에 있어서 타인자본 조달 및 자기자본 조달 모두가 중요하다. 또한, 두 가지 방법으로 지분을 보유할 수 있다: 유동화 및 권리 획득. 권리 획득을 통한 지분 보유의 경우, 단기 매매가 불가능한 장기 보유를 전제로 한다.

            \paragraph{}
                Steem 네트워크에서는 이러한 유형의 자산들을 다음과 같이 명명하고 있다: Steem (STEEM), Steem Power (SP) 및 Steem Dollars (SBD).

        \subsection{Steem (STEEM)}

            \paragraph{}
                Steem 블록체인의 기본 거래 단위는 Steem이다. 그 외의 모든 토큰들의 가치는 STEEM 가치를 기본으로 산정된다. 

            \paragraph{}
                STEEM은 유동성 화폐이므로 거래소에서 사거나 팔 수 있으며, 다른 사용자에게 지불의 형태로 송금할 수 있다.
        \subsection{Steem Power (SP)}

            \paragraph{}
                벤처 기업들은 장기 투자자들을 필요로 한다. 벤처 기업 투자자들은 장기 투자 후 이익을 실현하려고 한다. 장기 투자자를 확보하지 못한 벤처 기업이 유상 증자를 하려면 투자를 회수하려는 기존 주주들과 경쟁해야 할 수 있다. 합리적인 투자자들은 본인들의 투자를 통해 피투자 회사가 성장하기를 원한다. 하지만, 조달된 자금이 투자를 회수하려는 기존 주주들에게 분배되면 그러한 성장은 불가능하게 된다.

            \paragraph{}
                커뮤니티가 장기 계획을 수립할 수 있다는 점에서 장기 투자는 가치가 매우 높다. 또한, 주주들은 장기 투자를 통해 단기 실적보다는 장기적인 성장에 의결권을 행사할 수 있다.

            \paragraph{}
                암호화폐 분야의 경우, 투기자들은 단기 실적을 기준으로만 암호화폐를 선택한다. Steem은 장기적인 안목을 가진 구성원들로 구성된 커뮤니티를 구축하고자 한다.

            \paragraph{}
                사용자는 자신의 STEEM에 대해 13주의 장기 투자계획을 수립하여 플랫폼의 추가 혜택을 누릴 수 있다. 13주의 장기 투자에 사용된 STEEM을 Steam Power(스팀 파워; SP)라 한다. SP 잔액은 자동 반복 전환 요청을 통하지 않고는 송금할 수도 나눌 수도 없다. 이것은 SP가 암호화폐 거래소에서 쉽게 거래될 수 없음을 뜻한다.

            \paragraph{}
                사용자가 콘텐츠에 투표할 때 보상 풀에 대한 영향력은 가지고 있는 SP의 양에 직접 비례한다. 더 많은 SP를 보유한 사용자는 보상 배포에 더 많은 영향을 준다. 즉, SP는 Steem 플랫폼에서 독점 권한을 부여하는 엑세스 토큰이다.

            \paragraph{}
                SP 보유자는 남아있는 SP 잔액에 이자를 지급받는다. 연간 발행량의 15\%는 SP 보유자에게 이자로 지급된다. 이자로 받는 금액은 모든 사용자의 SP 총합 대비 보유 SP에 직접 비례한다.

            \paragraph{}
                STEEM을 SP로 전환하는 것을 "액면 병합(power up)"이라 하며, SP를 Steem으로 전환하는 것을 "액면 분할(power down)"이라 부른다. 액면 분할한 SP는 13주에 걸쳐 균등하게 반환되며, 분할 시점 1주 후부터 지급된다.

        \subsection{Steem Dollars (SBD)}

            \paragraph{}
                안정성은 글로벌 경제의 성공에 있어 매우 중요한 요소이다. 안정성 없이는 상업/저축 활동에 참여하는 개인들이 낮은 인지적 비용을 가질 수 없다. 이러한 안정성의 중요성으로 인해, 암호화폐와 Steem 네트워크 사용자들의 안정성이 Steem Dollars 설계에 반영되었다.

            \paragraph{}
                Steem Dollars는 벤처 기업들이 자금 조달 시 자주 활용하는 전환 사채와 유사한 구조를 갖는다. 벤처 기업에게 전환 사채는 향후 결정될 비율을 기준으로 주식으로 전환 가능한 단기 차입금에 해당한다. 블록체인 기반 토큰은 해당 커뮤니티에 대한 지분으로 간주될 수 있으며 전환 사채는 기타 상품 혹은 화폐로 표시된 차입금으로 간주될 수 있다. 전환 사채 투자자는 보유 전환 사채를 토큰의 공정 시장가격을 기준으로 손쉽게 예비 토큰으로 전환할 수 있다.

            \paragraph{}
                토큰-전환-달러의 거래를 통해 블록체인의 효율성뿐 아니라 토큰 투자자들의 투자 수익도 극대화할 수 있다. Steem Dollars를 약어 “SBD”로 표기할 수 있다. SBD를 발행하려면 믿을 만한 가격 자료, 악용 방지 규칙 및 유동성이 전제되어야 한다. 믿을 만한 가격 자료를 제공하려면 다음과 같은 세 가지 요인들이 충족되어야 한다: 부정확한 자료의 악영향 최소화, 부정확한 자료의 제공 비용 최대화, 시점의 중요성 최소화.

            \subsubsection{사기성 자료의 최소화}

                \paragraph{}
                    SP 투자자들은 가격 자료 제공자들을 선택한다. 아마도 가격 자료의 품질에 이권을 갖고 있는 사람들은 자료 제공자들을 신뢰할 것이다. Steem은 자료 제공자들에게 대가를 지불함으로써 자료 제공 권리를 획득하기 위해 자료 제공자들이 경쟁하는 환경을 조성한다. 대가가 클수록 허위 정보의 제공으로 인한 손해도 커진다.

                \paragraph{}
                    믿을 만한 자료 제공자들이 선택되었기 때문에, 전환에 적용되는 실제 가격 데이터가 제공될 수 있다. 이렇게 평판을 바탕으로 자료 제공자들이 관리되기 때문에 소수의 자료 제공자들이 신뢰성 있는 자료를 왜곡할 가능성이 낮아진다.

                \paragraph{}
                    모든 자료 제공자들이 정직하다고 할지라도, 통제 범위를 벗어난 사고로 인해 자료 제공자들이 악영향을 받을 수 있다. Steem 네트워크는 평균가격 자료의 단기적 왜곡을 견딜 수 있도록 설계되어 있으며, 커뮤니티는 그러한 왜곡을 교정하려고 노력한다. 교정이 오래 걸리는 한 예로 단기적 시세 조작을 들 수 있다. 난이도와 비용 측면에서 시세 조작을 오래 유지하는 것은 거의 불가능하다. 또 다른 예로 중앙 거래소의 실수 혹은 거래소 제공 정보의 왜곡을 들 수 있다.

                \paragraph{}
                    Steem은 3.5일 간의 평균가격을 이용함으로써 단기적인 가격 변동 요소들을 제거한다. 평균가격 자료는 1시간마다 업데이트된다.

                \paragraph{}
                    가격 자료 왜곡이 이동평균 시간대의 절반 동안만 지속되는 한, 전환 가격에 미치는 영향은 미미하다. 자료에 왜곡이 발생한 경우, 실제 전환 가격에 영향을 미치기 전에 네트워크 구성원들이 해당 자료의 제공자들을 투표로 몰아낼 것이다. 더욱 중요한 것은, 가격에 영향을 미치기 전에 잘못된 자료를 자발적으로 수정할 기회를 자료 제공자들에게 제공한다는 것이다. 매 3.5일마다 1.5일 간의 이슈 대응 기간이 커뮤니티 구성원들에게 주어진다.

            \subsubsection{타이밍 어택의 최소화}

                \paragraph{}
                    시장 참여자들은 블록체인의 3.5일 이동평균 전환 가격의 반응 속도보다 빠르게 정보에 접근할 수 있다. 커뮤니티의 비용으로 생성된 그러한 정보가 거래자들에게 혜택을 제공한다. STEEM 가치가 급격히 상승할 경우, 거래자들은 예전의 낮은 가격을 기준으로 SBD를 전환한 후 STEEM을 높은 가격으로 매도함으로써 리스크를 최소화할 수 있다.

                \paragraph{}
                    Steem은 모든 전환 요청들을 3.5일 연기하는 방법으로 시장을 안정화시킨다. 즉, 거래자 혹은 블록체인이 전환 시 적용되는 가격과 관련된 정보를 미리 입수할 수 없다.

            \subsubsection{전환 악용의 최소화}

                \paragraph{}
                    자유로운 양방향 전환이 허용될 경우, 거래자들은 블록체인 전환율을 이용하여 가격 변동 없이 대규모 거래를 실행할 수 있다. 가격 급등을 예상하는 거래자들은 높은 가격에서 (리스크가 가장 높을 때) SBD로 전환한 후, 교정 후 다시 전환할 것이다. Steem은 SBD → STEEM 전환만을 허용함으로써 그러한 유형의 악용을 원천 봉쇄하고 있다.

                \paragraph{}
                    블록체인은 SBD 발행 방법/시점/대상을 결정한다. 이를 통해 SBD 발행 비율을 안정적으로 유지할 뿐 아니라 악용도 예방할 수 있다.

            \subsubsection{지속 가능한 부채 비율}

                \paragraph{}
                    토큰 1개를 전체 토큰 조합의 지분으로 간주할 경우, 토큰-전환-달러는 부채로 간주될 수 있다. 부채 비율이 너무 높아지면, 전체 화폐가 불안정해진다. 부채 전환으로 인해 토큰 공급량이 급증하면서 가격이 하락할 수 있다. 그 다음의 전환은 더 많은 토큰들의 발행을 필요로 한다. 제대로 확인도 못한 상태에서 방대한 부채로 인해 시장이 붕괴할 수 있다. 부채 비율이 높을수록 신규 투자자들의 투자 의향은 떨어진다.

                \paragraph{}
                    STEEM 가치의 급격한 변동은 부채 비율의 급격한 변동으로 이어질 수 있다. 부채 비율이 10\%를 초과할 경우 블록체인은 SBD로부터 전환되는 STEEM의 양을 줄여 부채비율이 너무 높아지는 것을 방지한다. SBD 부채가 STEEM 시가 총액의 10\%를 초과할 경우 블록체인은 전환을 통해 발생하는 STEEM의 양을 시가 총액의 최대 10\%까지 자동으로 축소한다. 이를 통해 블록체인의 부채 비율은 10\%를 넘지 않는다.

                \paragraph{}
                    STEEM 발행량 계산에 사용되는 비율은 발행된 모든 SBD/SP들의 STEEM 가치를 포함한 공급량을 기반으로 한다 (현재 비율/공급량을 기준으로 결정됨).

                \subsubsection{이자}

                \paragraph{}
                    SBD는 투자자들에게 이자를 지급한다. 금리는 가격 자료를 발표한 동일한 사람들에 의해 설정되기 때문에 변화하는 시장 상황에 맞게 조정 가능하다. 모든 부채는 리스크를 수반한다. 전환 없이 SBD를 보유하고 있는 것은 커뮤니티에 1 달러를 빌려주는 것과 같다. 이는 미래에 1 달러에 SBD를 매수할 사람들이 존재하거나 전환 STEEM을 매수할 투기자들과 투자자들이 존재한다는 믿음을 기저로 한다.

                \paragraph{}
                    커뮤니티 구성원들이 SBD를 보유할 의향이 있다면 STEEM 및 SP 투자자들에게 레버리지 효과가 발생한다. 이러한 레버리지 효과는 성장에 기여할 뿐 아니라 성장에 따른 투자 이익도 확대시킨다. 가격 하락 시, STEEM 투자자들의 투자 가치가 희석된다. 암호화폐 프로젝트를 통해 Steem 네트워크에 출자한 사용자 기반의 확대로 인해 창출되는 Steem 네트워크의 부가가치가 희석 효과보다 크다는 것을 알 수 있었다.

            \subsubsection{가격 자료 결정}

                \paragraph{}
                    영리한 독자들은 공급이 제한적인 이자 발생 자산은 동일한 자산에서 얻을 수 있는 이자 조건에 따라 기초 자산보다 높거나 낮게 거래될 수 있다는 사실을 알게 될 것이다. 미국 달러 환율에 고정된 자산에 붙는 높은 금리로 인해 많은 사람들은 \$1 가치가 변동될 때까지 공급이 제한적인 Steem Dollars의 가치를 경쟁적으로 올리게 된다. 경제학에는, 아래 세 가지를 동시에 쟁취할 수 없다는 삼위일체 불가론\footnote{The Impossible Trinity, economic theory
\newline\url{https://en.wikipedia.org/wiki/Impossible_trinity}}이라는 경제 이론이 있다:

                \begin{enumerate}
                    \item 안정적인 환율
                    \item 자유로운 자본 이동
                    \item 독자적 화폐 정책
                \end{enumerate}

                \paragraph{}
                    Steem 자료 제공자들이 Steem Dollars 체계를 붕괴시키는 동시에 금리를 완벽히 통제할 수 있는 독자적 화폐 정책을 갖는 것을 목표로 할 경우, 곤란한 문제에 직면하게 될 것이다. 삼위일체 불가론에 따르면, Steem Dollars는 적어도 다음 요건들 중 하나를 충족시켜야 한다: 자본 이동의 제한; 불안정한 달러 환율; 제한적인 금리 통제.

                \paragraph{}
                    Steem 자료 제공자들의 주요 관심사는 SBD ~ 미국 달러(USD) 전환 비율을 안정적으로 1:1로 유지하는 것이다. SBD가 지속적으로 \$1.00 USD 이상에서 거래될 경우, 이자 지급이 즉시 중단되어야 한다. 0\% 금리에 여전히 할증이 요구되는 시장의 경우, 부채보다는 신용 확대의 경향이 나타나게 된다. 이러한 상황이 발생할 경우, 1 SBD의 가치가 \$1.00보다 높아지면서 마이너스 금리가 불가피해진다.

                \paragraph{}
                    부채 비율이 10\% 미만이고 SBD가 \$1.00 미만에서 거래될 경우, 금리를 인상해야 한다. 이를 통해 많은 사람들이 SBD에 투자할 수 있도록 장려하여 가격을 유지할 수 있다.

                \paragraph{}
                    SBD가 \$1.00 USD 미만에서 거래되고 부채 비율이 10\%를 초과할 경우, SBD 가치를 상향 조정해야 한다. 이를 통해 SBD 수요를 증가시키는 동시에 부채 비율을 낮춰 SBD ~ USD 전환 비율을 1:1로 맞출 수 있다.

                \paragraph{}
                    STEEM 가치의 상승 속도가 신규 SBD 발행 속도보다 빠를 경우, 부채 비율을 목표치 이하로 유지하여 모든 사람들에게 이자 혜택을 주어야 한다. STEEM 가치가 정체 혹은 하락할 경우, 어떠한 이자 지급도 부채 비율을 더욱 악화시킬 것이다.

                \paragraph{}
                    실제로, 자료 제공자들은 안정적인 USD 고정 환율을 유지하기 위하여 화폐 정책을 수립할 권한을 갖고 있다. 이러한 권한을 남용하면 STEEM 가치가 훼손될 수 있으므로, SP 투자자들은 위에 언급된 절차에 따라 가격/금리 자료 제공자들을 신중히 선택해야 한다.

                \paragraph{}
                    부채 비율이 너무 높아지고 시장 참여자들이 전환을 요청하지 않을 경우, SBD에 대한 STEEM 전환 비율을 상향 조정해야 한다.

                \paragraph{}
                    금리 정책 혹은 STEEM/SBD 전환율 할증/할인의 변경은 점차적으로 이뤄져야 하며, 단기가 아닌 장기 시장 상황에 맞게 평가되어야 한다. 블록체인은 단기 수요 흡수에 대한 대가를 유동성 공급자들에게 지불하고 있다.

                \paragraph{}
                    당사는 전술한 규칙들로 인해 \$1.00에 매수한 SBD를 보유해도 손실을 보지 않을 것이라는 믿음을 시장 참여자들에게 줄 수 있을 것으로 확신하고 있다. 또한, 당사는 SBD 가격이 \$0.95 ~ \$1.05 수준을 유지할 것으로 전망하고 있다.

        \subsection{개인 기여도}

            \paragraph{}
                주관적 작업 증명 시스템은 마이닝과 같은 \textit{객관적} 작업 증명 시스템을 대폭 보완한 화폐 분배 방식이다. \textit{주관적} 작업 증명을 통한 화폐 정책은 명확한 목표를 갖는 모든 커뮤니티들에 적용 가능하기 때문에 \textit{객관적} 작업 증명을 통한 화폐 정책보다 훨씬 더 광범위하다. 특정 커뮤니티에 가입한다는 것은 해당 커뮤니티를 신뢰하고 해당 커뮤니티의 성장을 위해 투표에 기꺼이 참여한다는 것을 의미한다.

            \paragraph{}
                실제로, 작업 평가 기준은 완전히 주관적으로 소스 코드에 의한 객관화가 불가능하다. 아티스트, 시인, 코미디언 등을 보상하는 데 우선 순위를 두는 커뮤니티가 있는 반면, 자선 활동이나 정치 이슈에 집중하는 커뮤니티들도 존재할 수 있다.

            \paragraph{}
                각 화폐의 가치는 특정 커뮤니티에 미치는 영향력과 각 커뮤니티의 성장 잠재성에 의해 좌우된다. 기존의 시스템들과는 달리, 주관적 작업 증명 시스템은 집단적인 가치 투자를 활성화할 뿐 아니라 비화폐성 자산들의 화폐화도 가능케 한다.

            \subsubsection{화폐 분배}

                \paragraph{}
                    \textit{매입} 혹은 \textit{가입}을 통해 암호화폐 커뮤니티에 참여할 수 있다. 두 경우 모두, 사용자들은 화폐와 관련하여 부가가치를 창출하고 있다. 하지만, 대부분의 사람들은 \textit{여분의 현금}보다는 \textit{여분의 시간}이 더 많다. 현금 \textit{자산}은 없고 \textit{시간}만 많은 열악한 커뮤니티에 화폐 제도를 도입한다고 상상해보자. 서로간의 거래를 통해 돈을 벌 수 있다면, 공정한 회계/화폐 제도를 기반으로 그러한 거래가 활성화될 수 있을 것이다.
    
                \paragraph{}
                    최대한 많은 사람들에게 공정하게 화폐를 분배하는 일은 매우 어려운 도전과제이다. 객관적 컴퓨터 알고리즘을 기반으로 평가 가능한 과제들이 제한적이기 때문에 외부 수익도 한정될 수 밖에 없다. Bitcoin 방식 마이닝의 경우, 전용 하드웨어가 필요할 뿐 아니라 좀 더 효율적인 알고리즘들의 개발에 많은 시간을 투자해야 한다. 이러한 노력이 복잡한 연산에 도움을 될 수 있겠지만, 화폐를 보유한 커뮤니티에 별다른 부가가치를 창출하지는 못한다. 더욱 중요한 것은, 규모의 경제 및 시장의 힘에 따라 전문가들을 제외한 사람들은 화폐 분배에 참여할 수 없게 된다. 궁극적으로, 연산 기반 마이닝은 전기 요금을 납부하기 위한 현금이나 관련 하드웨어의 개발을 필요로 하기 때문에 단순한 \textit{매입} 수단에 지나지 않는다.

                \paragraph{}
                    모든 사람들에게 공평하게 돈을 벌 수 있는 기회를 주려면, 먼저 참여의 기회를 주어야 한다. 여기서 도전과제는 수많은 사용자들의 참여의 질과 양을 어떻게 평가하여 그에 상응하는 보상을 제공하냐는 것이다. 이를 위해 확장 가능한 투표 프로세스가 요구된다. 특히, 자금 분배 권한을 최대한 분산하는 것이 필요하다.

                \paragraph{}
                    보상의 첫 번째 단계는 사용자들의 참여 혹은 투표 실적과 관계없이 고정된 화폐 분배 금액을 결정하는 것이다. 이 단계를 통해 관점이 \textit{"지급 여부"}에서 \textit{"지급 대상"}으로 바뀌며, "참여도"가 가장 높은 사용자들이 가장 많은 보상을 받고 있다는 사실을 시장에 암시하게 된다. 이는 가장 난해한 해시를 발견한 사람에게 50 BTC를 보상하는 Bitcoin과 유사한 구조다. Bitcoin과 마찬가지로, 모든 참여는 지급 이전에 이뤄져야 하며, 미래에 참여하겠다는 약속에는 어떠한 보상도 제공되어서는 안 된다.
    
                \paragraph{}
                    다음 단계는 \textit{참여도}가 가장 큰 사람들부터 보상하는 것이다. 참여도를 평가하고 그에 상응하는 보상을 제공한다. 경쟁이 치열할수록 보상을 받기가 그 만큼 어려워진다.

            \subsubsection{화폐 분배에 대한 투표권}

                \paragraph{}
                    분배할 자금이 고정 금액이고 장기 기득권을 가진 사람들에게 자금 분배에 대한 의사 결정 권한이 있다고 가정해보자. 투표권이 있는 모든 사용자들이 참여도 투표에 권리를 행사하고 나면, 당일 가용 자금이 투표 수를 기준으로 한 표라도 행사한 사용자들에게 분배된다.

                \paragraph{}
                    단순한 투표 프로세스로 인해 대의보다는 사익을 위해 본인에게 직접 투표하는 N-명 죄수의 딜레마 현상\footnote{N-명 죄수의 딜레마
                    \newline\url{https://cs.stanford.edu/people/eroberts/courses/soco/projects/1998-99/game-theory/npd.html}}이 발생할 수 있다. 모든 투표자들이 본인에게만 투표할 경우, 어떠한 화폐도 분배될 수 없으며 화폐 제도는 실패로 끝나고 만다. 한편, 투표자 1명만이 본인에게 투표한 경우, 화폐 가치의 변동 없이 해당 투표자가 공정하지 않은 이익을 얻게 된다.
    
                \subsubsubsection{부적절한 투표}

                    \paragraph{}
                        사용자 1명이 보유하고 있는 현금이 아무리 많을지라도, 그와 유사한 현금을 보유하고 있는 사람들이 언제나 있기 마련이다. 일반적으로, 부자 1등의 부는 부자 2등과 3등의 부를 합친 것보다 낮다. 또한, 커뮤니티에 많이 투자한 사람들일수록 투표 조작을 통해 잃을 것이 많다. 이는 마치 전사 이익을 갈취하기 위해 월급 지급을 중단하기로 결정한 CEO와 다름 없다. 직원들이 모두 회사를 떠나며 회사의 가치가 곤두박질치면서 CEO는 결국 파산하게 된다.
                    
                    \paragraph{}
                        다행히도, 투표가 많을수록 관심도가 높다. \textit{반대 투표}를 통해, 소액 주주들은 담합에 연루된 투표들을 무효화할 수 있다. 또한, 대주주들은 투표 조작을 통해 얻을 수 있는 것보다 화폐가치 하락으로 인해 잃을 것이 더 많다. 실제로, 반대 투표권을 행사하여 악용 행위를 방지하는 데 있어서 정직한 대주주들의 역할이 가장 크다.

                    \paragraph{}
                        악용 방지를 위한 \textit{반대 투표권의 행사}는 다른 사람들의 희생으로 한 사람만 이익을 보고 있다고 생각하는 \textit{크랩 멘탈리티(Crab Mentality)}를 최대한 활용한다. 일반적으로 크랩 멘탈리티란 근시안적인 사람들이 선량한 사람들을 깎아 내리는 것을 의미하지만, 선량한 사람들이 악한 사람들을 깎아 내리는 경우에도 적용되는 용어이다. 크랩 멘탈리티의 유일한 "문제점"은 다른 사람들의 희생으로 한 사람만 이익을 보고 있다고
\textit{오판}하는 것이다.

                \begin{quote}

                \subsubsubsection[크랩 멘탈리티 이야기]{The Story
of the Crab Bucket\protect\footnote{The Story of the Crab Bucket,
\url{http://guidezone.e-guiding.com/jmstory_crabs.htm}}}

                    \paragraph{}
                        바닷가를 걷고 있던 한 남자는 미끼통을 옆에 놓고 바닥 낚시를 하고 있는 남자를 보게 되었다. 가까이 가자, 뚜껑이 열린 미끼통 안에 살아있는 게들이 있는 것을 보았다.

                    \paragraph{}
                        남자는 물었다: "게들이 도망가지 못하도록 미끼통 뚜껑을 덮어야 하지 않나요?"

                    \paragraph{}
                        낚시꾼은 답했다: "뭘 잘 모르시네요. 게 한 마리만 있으며 바로 도망가겠지요. 하지만, 게 여러 마리가 들어 있으면 게 한 마리가 도망가려고 하면 다른 게들이 도망 가지 못하도록 끌어당겨서 결국 한 마리도 도망가지 못하고 운명을 같이 하게 되요."

                    \paragraph{}
                        사람들도 마찬가지다. 한 사람이 새로운 것을 시도하거나, 무언가를 개선하거나, 환경을 변화시키거나, 혹은 원대한 꿈을 꾼다면, 다른 사람들이 가만히 보고만 있지 않을 것이다.

                    \end{quote}

                    \paragraph{}
                        "악용"을 완전히 없애는 것은 불가능하므로, 목표로 설정될 수 없다. 그러한 "악용"을 시도하는 사람들조차도 여전히 커뮤니티 구성원들인 것이다. 성공적인 악용 혹은 담합으로 받게 되는 보상도 전형적인 Bitcoin 마이닝 혹은 담합적 마이닝 절차의 일환으로 인정될 수 있다. 중요한 것은 악용이 너무 만연해서 커뮤니티 및 화폐와 관련된 인센티브 제도가 훼손되지 않도록 하는 것이다.

                    \paragraph{}
                        커뮤니티 화폐를 발행하는 목표는 "미끼통의 게들"을 더 많이 담는 것이다. 모든 악용을 완전히 제거하기 위해 극단의 조치를 취한다는 것은 게 몇 마리가 도망가는 것을 막기 위해 미끼통의 뚜껑을 닫아 새로운 게들이 추가되는 것이 어려워지는 것과 같다. 벽을 미끄럽게 해서 게들이 서로 도망가지 못하게 하는 것만으로 충분하다.

            \subsubsection{배당금 분배}

                \begin{wrapfigure}{r}{0.5\textwidth}
                    \centering
                    \includegraphics[width=0.5\textwidth]
{img_the_new_marketplace}
                \end{wrapfigure}

                \paragraph{}
                    Steem 보상 체계의 주요 목적은 인터넷 상에서 건설적인 토론을 장려하는 것이다. 매년 Steem 발행 화폐의 75\%가 콘텐츠 생성/추천/댓글 사용자들에게 분배된다. Bitcoin 단위로 환산하면, 매일 수백만만 달러에 해당하는 금액이다.

                \paragraph{}
                    실제 배당금 분배는 사용자들의 투표 방식에 따라 달라지지만, 대부분의 보상은 가장 인기가 많은 콘텐츠에 분배되는 것으로 파악되고 있다.

              \paragraph{}
                    Zipf의 법칙\footnote{Zipf's Law
\url{https://en.wikipedia.org/wiki/Zipf\%27s_law}}은 다양한 현실세계 현상들을 특성화하는 경험적 규칙들 중 하나이다. Zipf의 법칙에 따르면, 규모 혹은 인기를 기준으로 대량 주문하면 두 번째 및 세 번째 항목이 미치는 영향은 각각 첫 번째 항목이 미치는 영향의 1/2 및 1/3에 지나지 않는다고 한다. 일반적으로, k번째 항목이 미치는 영향은 첫 번째 항목이 미치는 영향의 1/k이 된다.

                \paragraph{}
                    기여도 평가 항목으로 인기를 택한 경우, 각 개별 항목의 기여도는 Zipf의 법칙에 따라 결정된다. 즉, 백만 항목들이 있을 경우, 가장 인기가 높은 100개 항목들이 전체 기여도의 1/3을, 그 다음 10,000개 항목들이 두 번째 1/3을, 그리고 나머지 989,900 항목들이 마지막 1/3을 차지하게 된다. n개 항목들의 전체 기여도는 log( n )에 비례한다.

                \paragraph{}
                    이러한 투표 및 배당금 분배를 통해 양질의 콘텐츠에 대해 많은 보상을 제공하는 동시에 틈새 콘텐츠들도 보상하게 된다.

                \paragraph{}
                    롱테일 경제 효과는 표를 얻을 가능성을 과대 평가하여 보상보다 더 많은 노력을 기울이는 복권 추첨과 유사하다고 할 수 있다. 누구나 "무언가는 딴다"는 사실은 카지노들이 도박하는 사람들에게 심리학적으로 접근하는 방식이다. 다시 말해서, 작은 보상을 통해 많은 보상의 꿈을 심어주는 것이다.

                \subsubsubsection{배당금}

                    \paragraph{}
                        게시글에 배당금 지급 시, SBD 50\% 및 SP 50\% 형태로 지급된다. Steem Power는 투표권 및 협상력의 향상을, SBD는 안정적인 화폐 혜택을 사용자에게 제공한다. 앞에서 이미 논의한 바와 같이, SP와 SBD 모두 단기 매도보다는 장기 보유를 장려하도록 설계되었다. 이 장려책은 사용자들이 플랫폼의 장기적 성공에 관심을 가지도록 한다.

                    \paragraph{}
                        사용자들은 100\% SP를 받는 옵션을 선택할 수도 있다. 물론 배당금을 받지 않을 수도 있다. 사용자가 게시글의 배당금을 받지 않을 경우 보상금 풀에 남겨진 해당 금액은 다른 사용자들에게 분배된다.

    \section{합의 알고리즘}

        \paragraph{}
            합의란 커뮤니티 내에서 특정 정보에 대해 보편적인 합의에 이르는 과정을 말한다. 이미 다양한 합의 도달 알고리즘들이 개발되어 있다. 정부는 국민들이 헌법에 명시되어 있는 특정 규정들을 준수하기로 합의한 원시적인 합의 알고리즘이라 할 수 있다. 정부는 주관적 사실들을 해석하여 최종 결정을 내리기 위해 법원, 판사 및 배심원 제도를 구축한다. 대부분의 경우, 잘못된 결정이라 할지라도 사람들은 그러한 결정을 따른다.

        \paragraph{}
            암호화폐들에 적용되는 알고리즘들은 좀 더 효율적인 합의 도달 방법을 제시한다. 개인들의 암호화 서명된 증거들이 공공 거래장부에 기록된다. 그런 다음, 결정론적 컴퓨터 알고리즘이 공공 거래장부를 처리하여 보편적인 결론을 도출한다. 커뮤니티 구성원들이 처리 알고리즘에 합의하는 한, 알고리즘 처리 결과는 유효하다.

        \paragraph{}
            주요 고려사항은 공공 거래장부에 입력할 증언을 결정하는 것이다. 관련 시스템들은 검열 부담을 최소화할 수 있도록 설계되어야 한다. 공공 거래장부에 대한 검열은 특정인의 투표를 특정 선거에서 배제하는 것과 유사하다. 두 사례 모두에서, 개인들은 보편적 합의 과정에서 배제된다.

        \subsection{Steem의 합의 알고리즘}

            \paragraph{}
                개념상으로, Steem이 채택하고 있는 합의 알고리즘은 전세계 많은 기업들이 채택하고 있는 합의 알고리즘과 유사하다. Steem의 미래 가치에 대해 기득권을 가진 사람들이 투표를 통해 공공 거래장부에 포함시킬 증언를 선정하는 사람들을 선택한다. 투표권은 각 개인의 기득권에 따라 분배된다.

            \paragraph{}
                암호화폐 분야에서, 공공 거래장부는 보통 \textit{블록체인}이라 불린다. \textit{블록}은 서명된 거래들로 구성된다.

            \paragraph{}
                Steem의 경우, 블록 생성은 라운드 방식으로 이뤄진다. 각 라운드에서 21명의 증인들이 선정되어 거래 블록들을 생성하고 서명한다. 이 중 20명은 찬성 투표를 통해, 1명은 컴퓨터 기반 작업 증명을 통해, 그리고, 마지막 1명은 공동으로 선정된다. 본인이 생성한 블록들을 간과하는 것을 막이 위해 21명의 유효 증인들을 각 라운드에 교체 투입한다.

            \paragraph{}
                이러한 프로세스의 목적은 인기와 상관없이 모든 사람들이 블록 생성에 참여하도록 만드는 동시에 최상의 신뢰성을 제공하는 것이다. 사람들이 상위 20명의 선정된 증인들에 의한 검열에 대응하는 옵션은 세 가지가 있다: 1) 상위 20명에 포함되지 않은 사람들과 함께 인내심 있게 기다린다; 2) 다른 사람들보다 빨리 작업 증명을 처리할 수 있는 연산 능력을 확보한다; 3) 투표권 상승을 위해 더 많은 SP를 매입한다. 일반적으로, 검열은 선정된 증인들이 직업을 잃게 만드는 지름길이므로, Steem 네트워크 상에서 별다른 문제가 되지 않는다.

            \paragraph{}
                유효 증인들은 사전에 공개되기 때문에, Steem은 3초에 한번씩 증인들이 블록들을 생성하도록 만들 수 있다. 증인들은 NTP 규칙을 통해 본인들이 생성한 블록들을 동기화한다. 이 알고리즘을 약간 변형한 알고리즘이 BitShares에 의해 1년 넘게 사용되고 있으며 신뢰성이 검증되었다.

    \section{거래 수수료 면제}

        \paragraph{}
            Steem은 Steem 네트워크에 기여하는 사람들을 보상하는 데 중점을 두고 있다. 커뮤니티 참여 시마다 해당 구성원들을 교체하면 역효과를 낳을 수 있다.

        \paragraph{}
            현재, 블록체인 기술은 스팸 차단을 위해 거래 수수료에 의존하고 있다. 이러한 거래 수수료는 블록체인이 소액 거래에 활용되는 것을 방해하는 진입 장벽으로 작용한다. 분산 어플리케이션들이 중앙 집중식 대안들과 경쟁하려면 사용자들에게 무상 거래를 제공할 수 있어야 한다. 본 보고서에서는 수수료 면제를 통해 이전에 지원되지 않았던 분산 어플리케이션들의 광범위한 보급을 위해 Steem이 취하는 접근 방식을 소개한다.

        \subsection{수수료 관련 문제점}

            \paragraph{}
                블록체인은 모든 거래들이 동료들에게 중계되는 분산 네트워크이다. 미결 거래들의 일부 혹은 전부를 포함시키기 위하여 블록을 생성하는 경우가 종종 있다. 모든 블록체인들은 악의적인 사용자들이 쓸모 없는 거래들로 가용 네트워크 용량을 소모하는 것을 막을 수 있는 솔루션을 필요로 한다. 이러한 쓸모 없는 거래들은 다른 소중한 거래들이 처리되는 것을 방해하여 Steem 네트워크를 붕괴시킬 수 있다.

            \paragraph{}
                대부분의 블록체인들이 현재 채택하고 있는 솔루션은 최소 거래 수수료를 부과하는 것이다. 몇 센트 수준의 수수료도 Steem 네트워크의 수익성에 심각한 타격을 줄 수 있다. 이러한 접근 방식으로 스팸 문제는 해결이 가능할 수 있으나, 또 따른 문제를 야기한다. 이메일에 수수료를 부과하여 이메일 스팸 문제를 해결한다고 상상해보자; 아무도 이메일을 사용하지 않을 것이다.

            \subsubsection{소액 결제의 부적합성}

                \paragraph{}
                    거래 수수료 부과의 가장 핵심적인 문제는 사소한 사용자 서비스에 소액 결제가 적합하지 않다는 점이다. 모든 거래들에 수수료가 부과될 경우, 분산 네트워크가 처리할 수 있는 거래 유형들을 제한한다. 수수료 필요성 논리의 정당성과 상관없이, 사용자들은 여전히 소액 수수료에 강한 거부감을 느낄 것이다

                \paragraph{}
                    비밀번호를 변경할 때마다 수수료를 부과하는 웹사이트가 있다고 가정해보자. 사용자들은 어느 정도의 무료 서비스들을 기대한다. 소액 수수료를 사용자들에게 강요하면 걷잡을 수 없는 탈퇴로 이어질 것이다.

                \begin{quote}
                        거래 자체의 부가가치는 매우 낮아서 별도의 의사 결정이 요구되지 않아야 한다. 금액과 관계없이 구매 의사 결정은 신중함이 요구되는 프로세스이기 때문에, 관련 절차 혹은 소요 시간이 아니라 결정 프로세스 자체로부터 최종 결과가 도출된다.\\

                        다른 결제와 마찬가지로, 소액 결제도 비교 분석이 수반된다: "이 정도의 X 가치가 저 정도의 Y 가치에 맞먹는가?". 사용자가 아무런 망설임 없이 수락할 수 있는 거래는 오직 무료 거래이기 때문에, 언제나 심리적인 최소 거래 비용이 존재하기 마련이다.

                        - Clay Shirky\protect\footnote{Clay Shirky, The
Case Against Micropayments\newline
\url{http://www.openp2p.com/pub/a/p2p/2000/12/19/micropayments.html}}\\
                \end{quote}

                \paragraph{}
                    금융 결제 시장의 경우, 수수료보다 이미 구매한 거래의 가치가 훨씬 높기 때문에 소액 수수료에 대한 거부감이 적다. 블록체인 어플리케이션 시장의 경우, 금융 결제 시장과 달리 사용자들이 수수료에 대해 강한 거부감을 느끼는 거래들이 매우 많다.

                \paragraph{}
                    BitShares, Nxt, Ripple, Counter Party, Stellar 등은 모두 사용자들이 블록체인 상에서 주문을 제한할 수 있도록 허용하고 있으며, 그러한 프로세스에 대해 소액 수수료를 부과하고 있다. 주문 취소에 대해서도 추가 수수료가 부과된다. Ethereum 등은 완전히 다른 방식으로 소액 결제 수수료를 부과한다: 계산 건 별 수수료 부과. 이러한 시스템들은 모두 사용자 유치에 어려움을 겪고 있으며, 이는 검색에 소액 수수료를 부과하는 분산 검색 엔진이 Google로부터 사용자들을 뺏어 오는 데 어려움을 겪는 것과 매우 흡사하다. 서비스 질과 상관없이 사람들은 어느 정도의 무료 서비스를 기대한다. 이러한 사실은 결과적으로 더 많은 수수료를 내는 경우에도 여전히 유효하다.

            \subsubsection{진입 장벽으로 작용하는 수수료}

                \paragraph{}
                    종류와 상관없이 수수료는 신규 사용자들에게 진입 장벽으로 작용한다. Ethereum 서비스를 경험하려면 먼저 ETH 토큰을 매입해야 한다. Ethereum 상에서 분산 어플리케이션을 생성하려면 관련 비용을 다른 고객들에게 전가해야 한다. 암호화폐를 매입하는 것은 간단한 일이 아니므로, \$10 이하의 거래는 의미가 없다. 이는 \$10 수수료의 정당성을 새로운 분산 어플리케이션을 시도하려는 신규 사용자들에게 납득시켜야 함을 의미한다.

            \subsubsection{수수료 조정}

                \paragraph{}
                    시간 경과에 따라 네트워크는 수수료를 조정해야 한다. 이러한 상황은 토큰 가치의 상승 혹은 용량 급증으로 인해 발생할 수 있다. 사용자들은 예측 가능한 수수료와 안정적인 서비스를 원한다. 이용 폭증 시간대에 맞게 수수료를 조정할 수 있지만, 사용자 경험의 악화로 이어질 수 있다.

            \subsubsection{Sybil 공격}

                \paragraph{}
                    중앙 집중식 웹사이트들은 비율 제한 및 ID 인증을 통해 스팸을 차단한다. reCAPTCHA\footnote{reCAPTCHA, Easy on Humans, Hard on Bots\newline
\url{https://www.google.com/recaptcha/intro/index.html}}와 같은 단순한 방법으로도 허위 계정의 개설을 막을 수 있다. 누군가가 계정을 남용할 경우, 중앙 집중식 웹사이트들은 해당 계정을 차단할 수 있다.

                \paragraph{}
                    분산 시스템의 경우, 사용자들을 차단하거나 중앙 집중식 서비스 제공자가 reCAPTCHA를 통해 관련 계정들의 비율 제한 조치를 시행할 수 있는 직접적인 방법이 존재하지 않는다. 실제로, 사용자들을 차단할 수 없는 것이 블록체인 기술의 주요 장점 중 하나이다.

            \subsubsection{완전 유보 vs. 부분 유보}

                \paragraph{}
                    지역 회선을 독점하고 있고 언제라도 제공 가능한 최대 대역폭을 보유하고 있는 인터넷 서비스 제공자(Internet Service Provider; ISP) 협동조합과 같은 블록체인을 살펴보자. 해당 지역의 거주자들은 ISP 지분을 매입하여 원하는 만큼의 대역폭을 이용할 수 있다.

                \paragraph{}
                    ISP가 선택 가능한 옵션은 "완전 유보" 혹은 "부분 유보" 시스템을 운영하는 것이다. 완전 유보 시스템의 경우, 사용자 별로 지분에 비례하여 최대 대역폭의 일부만 허용된다. 모든 사용자들이 동시에 인터넷을 이용하지는 않기 때문에, 해당 지역의 네트워크는 충분한 유휴 공간을 갖게 된다.

        \subsection{대역폭 - 소액 결제 창구의 대안}

            \paragraph{}
                소액 결제와 관련된 문제를 해결하기 위한 솔루션은 \textit{동적인 부분 유보}의 구현에서 찾을 수 있다. 이 모델에서는, 네트워크 정체 시 블록체인에서 자동으로 유보율을 조정한다. 블록체인은 단기적인 수요 증가에 대응할 수 있는 목표 가동률을 설정한다. 수요 급증이 지속될 때마다, 블록체인은 최대 지분당 대역폭을 낮춘다. 수요 급증이 완료되어 잉여 용량이 존재할 경우, 블록체인은 지분당 대역폭을 서서히 올릴 수 있다.

            \paragraph{}
                개인 사용자가 이용하는 대역폭은 사용자가 이용 시간대를 조정할 수 있도록 충분히 넉넉한 기간에 걸쳐 측정되어야 한다. 일단 로그인한 사용자들은 너무 많은 것들을 한번에 처리하고 로그아웃 하려는 경향이 있다. 이는 단기적인 이용 대역폭이 장기적인 이용 대역폭에 비해 너무 높게 나타남을 의미한다. 이용 시간대가 너무 많이 연장될 경우, 단기 수요 급증에 맞게 유보율이 신속히 조정되지 않아서 다른 사용자들에게도 영향을 미치게 된다.

            \paragraph{}
                당사의 추정에 따르면, 사용자들의 주 평균 대역폭 사용량을 측정하는 것만으로 충분해야 한다. 사용자가 거래를 체결할 때마다, 해당 거래가 해당 사용자의 이동평균 값에 반영된다. 사용자의 이동평균 값이 기존의 네트워크 허용치를 초과할 경우, 허용치를 밑돌 때까지 해당 사용자의 거래가 지연된다.

                \subsubsubsection{용량의 영향}

                    \paragraph{}
                        블록체인 용량에 반드시 한도가 설정되는 것은 아니다. Bitcoin 블록 크기를 10MB까지 증가시킬 수 있는 인터넷 인프라가 구축되어 있기 때문에, 그러한 블록 크기 증가는 최소 필수 잔고의 10배 감소로 이어진다. Bitcoin은 현재 초당 거래 3건을 지원하고 있지만, 초당 거래 1,000건도 구현 가능하다.

                \subsubsubsection{수수료 비교}

                    \paragraph{}
                        \$25 가치의 BTC를 보유 중인 사용자가 주 1회 거래하고 수수료로 4 센트를 지급할 경우, 연간 수수료는 \$2.00을 초과한다. 수수료를 포함한 \$25 투자에 대해 손익 분기를 넘으려면 최소 8\%의 투자 수익률이 필요하다. 사용자들이 블록체인 투자 자금을 보유할 확률이 높기 때문에, \$25 가치의 BTC를 보유 중인 사용자는 수수료 기반 접근법 대신에 비율 제한 접근법을 통해 연간 \$2 정도를 절약하는 셈이 된다. \$175 만으로도 매일 거래가 가능하고 연간 \$14를 절약할 수 있다.

            \subsubsection{계정 개설}

                \paragraph{}
                    잔고가 공개되는 Steem의 계정 기반 시스템은 대역폭 기반 비율 제한 알고리즘의 구현을 간소화한다. 주 1회 거래에 필요한 최소 잔고를 밑도는 계정으로는 거래가 불가능하다. 즉, 모든 신규 계정들의 잔고는 최소 잔고 이상이어야 한다. 또한, 잔고가 충분하고 계정을 다시 사용하는 한, 언제라도 소액 거래가 가능하다.

                \paragraph{}
                    사용량이 낮을 때 개설된 잔고가 낮은 계정의 경우, 네트워크 사용량이 증가하면 접근이 불가능할 수 있다. 잔고를 보충하면 언제라도 복구가 가능하다.

                \paragraph{}
                    최소의 기존 계정들을 갖고 합리적인 사용자 경험을 유지하려면, 모든 신규 계정들의 잔고가 주 1회 거래에 요구되는 최소 잔고의 10배 이상이어야 한다. 이러한 방식으로 수요가 10배 증가하는 경우에도 해당 계정의 유효성을 유지할 수 있다.

                \paragraph{}
                    초기 계정 잔고는 Sybil 공격에 대비한 토큰 발행이 아니라 해당 계정을 개설한 사용자에 의해 입금되어야 한다.

            \subsubsection{최소 잔고의 적절성 검증}

                \paragraph{}
                    최소 잔고 유지 요건은 사용자 가치를 기준으로 결정된다\footnote{Forbes, Tristan
Louis, "How Much is a User Worth?"\newline
\url{http://www.forbes.com/sites/tristanlouis/2013/08/31/how-much-is-a-us}}. 사업을 하는 사람이라면 누구나 모든 사용자들이 매우 소중하다는 사실을 잘 알고 있을 것이다. 보통 기업들은 사용자 1명을 유치하는 데 \$30 ~ \$200 정도를 투자한다. 사용자들에게 직접 돈을 주거나 광고비를 지출하며, 심지어 사용자 기반을 확보하기 위해 기업을 인수하기도 한다. 사용자들을 확보하고 나면, 기업들은 고객 유지를 통한 미래 수익 창출을 위해 다양한 무료 서비스들을 제공한다.

                \paragraph{}
                    Ripple은 계정 별로 최소 잔고\footnote{Ripple, Account
Reserves\newline\url{https://ripple.com/build/reserves/}}를 운영하고 있으며, 모든 신규 계정들에 그러한 최소 잔고를 의무화하고 있다. 현재, Ripple의 최소 잔고는 \$0.15로, 앞에서 언급된 주 1회 거래에 요구되는 \$0.10보다 높다.

                \paragraph{}
                    블록체인은 간단한 최소 잔고 설정 프로세스를 통해 사용자 별로 최소 값을 설정할 수 있다. 신규 고객을 유치하기 위해 거래가 가능하도록 해당 고객의 계정에 최소 잔고를 미리 채울 수도 있다. 신규 가입 시 상대적으로 높은 수수료(\$1.00)을 부과한다면 다른 경쟁자들에게 고객들을 빼앗기는 결과를 초래할 수 있다.


                \paragraph{}
                    사용자 관점에서 보면, 최소 잔고를 유지한다는 것은 해당 잔고에 대한 이자와 함께 거래 수수료를 지출하는 것과 같다고 할 수 있다. 최소 잔고란 수수료를 상쇄하는 이자를 단기에 벌기 위해 필요한 잔고라 할 수 있다.

                \paragraph{}
                    다행히도, 필수 최소 잔고는 1 달러 수준으로 사용자들이 거부감을 느끼는 금액은 아니다. 손해 본 이자에 대한 기회 비용이 소액 수수료의 인지적 비용을 발생시키지 않기 때문에, 더더욱 사용자들의 거부감은 없게 된다.

                \paragraph{}
                    신규 계정의 잔고를 채울 때 사용되는 STEEM은 Steem Power로 액면 병합된다. 새 계정의 SP 잔고의 일부는 계정 생성자로부터 임대받을 수 있다. SP를 임대받은 사용자는 그들이 보유한 SP와 동일하게 투표 및 대역폭에 사용할 수 있다. 임대받은 SP의 소유권은 임대인에게 귀속된다. 사용자는 언제든지 임대를 끝낼 수 있다. 쿨다운 기간이 지나면 SP는 임대인 계정으로 돌아간다.

            \subsubsection{수수료 대비 유효성}

                \paragraph{}
                    비율 제한의 유효성을 수수료와 비교하려면, 두 시스템들이 어떻게 공격자에 의한 국제적인 네트워크 공격에 대응하는지를 고려해야 한다. Bitcoin의 경우, \$10,000를 보유한 공격자는 모든 개별 블록들을 채우며 하루 종일 서비스를 중단시킬 수 있다. 동적인 부분 유보율 제한 방식의 경우, 동일한 공격자가 개별 블록 1개의 서비스조차도 중단할 수 없을 것이다.

                \paragraph{}
                    공격자가 전체 코인의 1\%를 보유한다는 극단적인 가정 하에서, 공격자는 최대 6천만 달러까지 보유할 수 있게 된다. 마이너들이 수수료 혹은 용량을 인상하지 않는 한, 공격자는 Bitcoin 블록체인 서비스를 16년 동안 거부할 수 있다. 수수료가 거래 당 \$15까지 인상된다고 할지라도, 여전히 공격자는 16일 동안 네트워크를 공격할 수 있다.

                \paragraph{}
                    비율 제한 접근 방식의 경우, 전체 코인의 1\%를 보유한 사람이 네트워크를 공격하는 데 30초도 걸리지 않는다.


            \subsubsection{임대 vs. 구매 vs. 공동 소유}

                \paragraph{}
                    주택 소유자라면 누구나 소유 주택을 무료로 사용할 것을 기대한다. 공동으로 주택을 매입할 경우, 각각의 소유주는 본인 지분에 맞게 해당 주택을 이용할 것으로 기대한다. 수수료 기반 블록체인은 주택을 임대하는 것과 유사하며, 비율 제한은 소유주 간의 공동 소유와 같은 역할을 한다.

                \paragraph{}
                    주택이 복수의 소유주들에 의해 소유되는 경우, 해당 소유주들은 공동 소유 방법을 결정해야 한다. 지분이 50\%에 달하지만 1년에 주말 1회 사용하는 소유주는 본인 대신에 사용하는 소유주들로부터 보상을 받을 것으로 기대할 수 있다. 수수료 기반 시스템도 그러한 원리가 그대로 적용된다.

                \paragraph{}
                    반면, 주택 지분 50\%를 보유한 사람은 주택 수요가 증가하여 더 높은 가격에 지분을 매도할 수 있을 것이라고 짐작할 수 있다. 실제로 사용하지 않고 높은 지분만 보유하고 있는 소유주는 부동산 투기자가 된다. 즉, 임대료가 목적이 아니라 부동산의 가치 상승을 노리는 것이다.

                \paragraph{}
                    지분의 가치는 해당 소유주가 얼마나 오래 권리를 누릴 수 있는지에 의해 좌우된다. 주택 지분 1\%를 보유한 사람이 1년에 주말 1회 해당 주택을 사용할 경우, 해당 지분의 가치는 매우 낮다고 할 수 있다. 단, 주주들의 절반이 전혀 주택을 이용하지 않을 경우, 공동 소유 가치는 연간 주말 2회로 상승한다. 이러한 유휴 사용자들이 미사용 시간을 임대하기로 결정할 경우, 공동 소유 가치는 다시 연간 주말 1회로 하락한다. 미사용 공동 소유권이 다른 사람들에게 매도된 경우, 공동 소유 가치는 50\% 하락한다. 임대료가 지분 가치의 하락분보다 크지 않는 한, 공동 소유주들은 경제적으로 손해를 보고 있는 것이다.

                \paragraph{}
                    위에 언급된 이유를 근거로, 수수료 기반 시스템은 사용자 관점에서는 비용이 많이 들거나 공동 소유주 관점에서는 수익성이 떨어진다고 가정할 수 있다. 개인 소액 소유주가 임대를 통해 이익을 얻을 수 있으나, 다른 모든 공동 소유주들의 희생이 있어야만 가능하다. 실제로, 공동 소유 가치의 하락에 따른 비용은 모든 주주들이 공동으로 분담하는 반면, 이익은 임대를 결정한 소유주 1명이 독차지한다.

                \paragraph{}
                    따라서, 사용료가 없을 때 블록체인이 가장 효율적으로 운영된다는 결론을 내릴 수 있다. 비율 제한의 대안으로 사용료를 부과할 경우, 공동 소유 가치에 맞먹어 소유주들이 해당 권리를 장기 보유할 수 있어야 한다.

                \paragraph{}
                    다시 말해, 거래 수수료가 주 1회 거래에 요구되는 최소 계정 잔고와 동일해야 하며, 매주 환불이 가능해야 한다. 최소 계정 잔고가 \$1이고 주 1회 거래가 허용된다고 가정해보자. \$1 잔고를 보유 중인 사람이 한번에 5건의 거래들을 실행하려고 할 경우, 해당 거래 이전 혹은 이후에 잔고를 \$5까지 올려야 한다.

                \paragraph{}
                    이론적으로, 사용자들이 필요한 지분을 매입할 수 있는 시장이 형성될 수 있다. 현실적으로는, 사용률에 맞게 공동 소유권을 매매하는 게 훨씬 효율적이다. 다시 말해서, 소액 대출의 협상과 관련된 비용이 최대 주간 사용량에 적합한 잔고를 유지하는 비용보다 높은 것이다.

                \paragraph{}
                    거래들의 분산된 비율 제한을 통해 소액 결제로 인해 사용이 불가능했던 새로운 유형의 분산 어플리케이션들을 사용할 수 있게 된다. 이러한 신규 모델은 어플리케이션 개발자들에게 거래 수수료 부과 여부 및 부과 시점을 결정할 수 있는 선택권을 준다.

    \section{성능 및 확장성}

        \paragraph{}
            Steem 네트워크는 BitShares에 적용된 기술인 Graphene을 기반으로 구축된다. Graphene은 공개 시연을 통해 분산 네트워크 상에서 초당 1,000건 이상의 거래들을 지원할 수 있다는 것을 입증하였다. 또한, 간단한 서버 용량 및 통신 프로토콜 개선을 통해 초당 10,000건 이상의 거래들까지 확장 가능하다.

        \subsection{Reddit의 규모}

            \paragraph{}
                Steem은 Reddit보다 광범위한 사용자 기반을 처리할 수 있다. 2015년 기준으로, Reddit의 870만 사용자들은 초당 23개의 댓글들을 생성하였으며\footnote{Reddit Statistics, Number of Users and Comments per
Second\newline\url{http://expandedramblings.com/index.php/reddit-stats/2/}}, 이는 사용자 1명당 연평균 댓글 83개에 해당하는 수치이다. 7,300만개의 상위 게시글들이 있었으며, 이는 초당 평균 2개의 신규 게시글에 해당한다. 대략 70억 개의 “좋아요” 투표들이 있었으며, 이는 초당 평균 투표율 220개에 해당한다. 결론적으로, Reddit이 블록체인 상에서 운영된다면 초당 평균 250건의 거래들이 필요한 것이다.
                
            \paragraph{}
                이러한 업계를 선도하는 성능을 달성하기 위해, Steem은 초당 600만 건의 거래들을 처리할 수 있는 LMAX 거래소\footnote{Martin Fowler, The
LMAX Architecture\newline\url{http://martinfowler.com/articles/lmax.html}}를 벤치마킹 했다. 벤치마킹을 통해 다음과 같은 규칙들을 도출하였다:

            \begin{enumerate}
                \item 모든 것을 메모리에 저장한다.
                \item 핵심 비지니스 로직을 단일 스레드에 저장한다.
                \item 암호화 작업(해시 및 서명)과 핵심 비지니스 로직을 분리한다.
                \item 검증을 상태 의존적 점검과 상태 독립적 점검으로 나눈다.
                \item 객체 지향형 데이터 모델을 활용한다.
            \end{enumerate}

            \paragraph{}
                위 규칙들을 준수함으로써, Steem은 별다른 최적화 작업 없이 초당 10,000건의 거래들을 처리할 수 있다.

            \paragraph{}
                Intel의 Optane™ 기술\footnote{Introducing Intel Optane Technology - Bringing 3D XPoint Memory
to Storage and Memory Products\newline
\url{https://newsroom.intel.com/press-kits/introducing-intel-optane-technology-bringing-3d-xpoint-memory-to-storage-and-memory-products/}}을 활용하면 모든 것을 매우 효율적으로 메모리에 저장할 수 있다. 모든 게시글들이 인덱스 메모리에 저장된 상태에서 범용 하드웨어가 Steem과 관련된 모든 비지니스 로직을 단일 스레드 상에서 처리할 수 있다. Google 조차도 전체 인터넷 인덱스를 RAM에 저장하고 있다. 블록체인 기술을 활용하면 데이터베이스를 많은 기기들에 손쉽게 복사하여 데이터 손실을 예방할 수 있다. Optane™ 기술이 진화하면서, 신뢰성 개선과 함께 RAM 속도는 더욱 빨라질 것이다. 다시 말해서, Steem은 미래의 아키텍처 및 확장성에 최적화되어 있다.

    \section{초기 분배 및 공급}

        \paragraph{}
            Steem 네트워크의 경우, 0부터 화폐 공급이 시작되며, 작업 증명을 통해 분당 40 STEEM 비율로 마이너들에게 STEEM을 분배한다. 또한, 콘텐츠 및 큐레이션 보상 조합들을 지원하기 위해 분당 40 STEEM을 추가로 분배한다 (분당 총 80 STEEM). 그런 다음, Steem 네트워크는 SP로 전환하는 사용자들을 보상하기 시작한다. 이 시점에서, 아래 요약된 다양한 기여도 보상 혜택들로 인해 STEEM은 분당 800 STEEM 비율로 증가한다:

        \paragraph{}
            \textbf{기여도 보상:}

        \begin{itemize}
            \item 큐레이션 보상: 블록 당 1 STEEM 또는 연간 3.875\% 중 높은 가치
            \item 콘텐츠 생성 보상: 블록 당 1 STEEM 또는 연간 3.875\% 중 높은 가치
            \item 블록 생성 보상: 블록 당 1 STEEM 또는 연간 0.750\% 중 높은 가치
            \item 블록 864,000 이전 POW 포함 보상: 블록 당 1 STEEM (라운드 당 21 STEEM 보상)
            \item 블록 864,000 이후 POW 포함 보상: 블록 당 0.0476 STEEM (라운드 당 1 STEEM 보상) 또는 연간 0.750\% 중 높은 가치.
            \item 유동성 보상: 블록 당 1 STEEM (awarded as 1200 STEEM per hour) 또는 연간 0.750\% 중 높은 가치
        \end{itemize}

        \paragraph{}
            \textbf{Steem Power 보상:}

        \begin{itemize}
            \item Steem Power 보상: 기여도 보상으로 발행된 STEEM 별로, 9 STEEM이 모든 Steem Power 투자자들을 대상으로 분배된다.
        \end{itemize}

        \paragraph{}
            \textbf{SBD 운영:}

        \begin{itemize}
            \item SBD 보상: SBD 가치의 일정 비율이 증인들이 설정한 APR를 기준으로 발행된 후, SBD 투자자들에게 SBD로 지급된다.

        \end{itemize}

        \paragraph{}
            총 공급량 개념은 후속 공급률과 SBD 보상을 통해 대규모 STEEM 창출 혹은 소멸로 이어지는 SBD 운영의 영향으로 더욱 복잡해진다 (SBD 단락 참조). 또한, 다음과 같은 추가적인 요인들이 영향을 미친다: 찾아가지 않은 인센티브 (예. 누락된 블록들에 대한 블록 보상), 버려진 계정.

    \section{현재 분배 및 공급}

        \paragraph{}
            2016년 12월 16번째 하드 포크부터 Steem은 매년 9.5\%의 인플레이션율로 새로운 토큰을 만들고 있다. 매 250,000 블럭 당 0.01\%의 인플레이션율 혹은 연간 0.5\%의 인플레이션율이 감소한다. 인플레이션율은 0.95\%에 다다를 때 까지 계속 감소한다. 16번째 하드포크 이후 20.5년의 시간이 걸릴 것이다.


        \paragraph{}
            만들어진 토큰 중 75\%는 보상 풀에 보관하며 이를 저자와 큐레이터가 나누어 갖는다. 15\%는 SP 보유자에게 지급한다. 남은 10\%는 블록체인을 구동하는 증인에게 지급한다.

        \subsection{토큰 발행률의 영향}

            \paragraph{}
                종종 인플레이션 모델을 가진 화폐는 지속 가능하지 않다고 말하지만, 다양한 실제 사례들을 통해 화폐 수량이 화폐 가치에 직접적이고 즉각적인 영향을 미치지 않는 것을 알 수 있다.
    
            \paragraph{}
                2008년 8월부터 2009년 1월까지, 미국의 통화 공급량\footnote{United States Money Supply, 2009\newline\url
{https://research.stlouisfed.org/fred2/graph/?s\%5B1\%5D\%5Bid\%5D=AMBNS}}은 8,710억 달러에서 1조 7,370억 달러로 늘어났으며, 이는 연평균 100\%가 넘는 증가에 해당한다. 이후 6년 동안 대략 연평균 20\% 수준으로 공급량이 늘어났다. 다시 말해, 미국의 통화 공급량이 4.59배까지 늘어나는 데 7년도 걸리지 않은 것이다. 미국 정부의 물가 지수\footnote{CPI Inflation Index, United States Dollar 2008-2016\newline
\url{http://data.bls.gov/cgi-bin/cpicalc.pl?cost1=1&year1=2008&year2=2016}}에 따르면, 동일 기간 동안에 달러의 상대적 가치는 10\% 이하로 하락했다. 이 사례를 통해 알 수 있듯이, 공급량은 가격의 한 요소에 지나지 않는다.

            \paragraph{}
                Bitcoin의 첫 2년간 연간 인플레이션율\footnote{Bitcoin Transaction Volume\newline
\url{https://blockchain.info/charts/estimated-transaction-volume}}은 100\% 이상을 유지했다. 처음 5년을 보면 30\% 이상이었고, 처음 8년을 보면 10\%를 넘는다. Steem이 콘텐츠, 큐레이션, 마이닝 및 유동성 보상을 위해 지출한 총 "비용"은 10\% APR에 지나지 않는다.

            \paragraph{}
                STEEM과 같은 디지털 상품의 가격은 공급과 수요에 의해 결정된다. 장기 투자자가 매도를 결정하면, 시장의 STEEM 공급량이 증가하며 가격을 하락시킨다. 이러한 하락 압박은 새로운 장기 투자자가 STEEM 매입 및 SP 전환 결정 시 상쇄된다. 미래 시장 가격 예측으로 유동성 STEEM을 매도/매수하는 투기업자로 추가적인 공급과 수요가 발생할 수 있다.
                
    \section{Steem의 위력}

        \paragraph{}
            Steem은 사용자 기여도(게시글 및 투표)의 가치가 매우 소중하다는 것을 잘 알고 있다. 댓글 1개도 나름대로의 가치를 가지지만, 수백만 개의 맞춤형 게시글들은 수백만 혹은 수십억 달러의 가치를 갖는다. 투표 1표가 갖는 큐레이션 가치는 거의 없지만, 수십억 표는 엄청난 큐레이션 가치를 갖게 된다. 큐레이션 없는 콘텐츠의 가치는 제한적이다. 연결 링크를 제외한 인터넷 상의 모든 콘텐츠를 고려했을 때, Google은 유용한 검색 결과를 제공하는 데 어려움을 겪고 있다. 실질적으로 높은 가치를 제공하는 것은 정보 간의 연결 링크들이다.

        \paragraph{}
            모든 사람들이 혜택을 입으므로, 모든 사람들이 일정 대가를 지불해야 한다. 다시 말해서, 구성원들은 Steem에 부가가치가 창출되는 것에만 대가를 지불하는 것이다. 당사가 할 일은 사용자 기여도가 소셜 네트워크에 부가가치를 창출하고 있는지 확인만 하면 되는 것이다. 종합적으로, Reddit 사용자들은 초당 220회를 투표하고 23 게시글들을 작성한다. Reddit의 시가 총액은 5억 달러\footnote{Reddit
Valuaton, Newsweek, 2014\newline
\url{http://www.newsweek.com/investors-think-reddit-worth-500-million-26}} ~ 40억 달러\footnote{Worth of Web, March 2016\newline
\url{http://www.worthofweb.com/website-value/reddit.com/}} 수준으로, 모든 투표들과 게시글들에 각각 $0.06 ~ $0.50 수준의 가치가 있음을 의미한다. Reddit 기업 가치의 대부분이 새로운 참여를 유발하는 전 주에 이뤄진 실시간 토론들에서 창출된다고 주장하는 사람이 있을 수 있다. 과거가 아닌 현재에 사람들이 많은 곳에 사람들이 몰리게 마련이다.

        \subsection{소액 결제 불가 (후원금 = 옵션)}

            \paragraph{}
                암호화폐를 소셜 미디어 플랫폼에 통합하려는 기존의 시도들은 사용자들 간의 결제 기능에 집중했으며, 많은 서비스들이 후원금 제도를 도입하려고 시도했다. 이론적으로, 후원금 제도가 간단할수록 더 많은 사람들이 후원 활동에 참여할 것이다. 콘텐츠 홍보비를 지불하는 것을 장려하는 서비스들도 있고, 기사가 벌어들일 수 있는 후원금 액수를 추정해주는 서비스들도 있다.

            \paragraph{}
                위에 언급된 모든 서비스들은 소액 결제를 필요로 하며, 유일한 차이는 지불 주체가 누구냐는 것이다. 해당 서비스들은 부족한 소액 결제 사용자들로 인해 어려움을 겪고 있다. 인센티브 기반 콘텐츠 제작을 추구하는 과정에서, 기업가들은 지불 주체에만 집중하며 극명한 사실을 간과하고 말았다: 모든 사람들이 서로 간의 거래로 인해 이익을 얻기 때문에 관점에 따라 모두 지불하거나 아무도 지불하지 않아야 한다.

            \paragraph{}
                사용자가 게시글에 “좋아요” 투표를 하면 지불 주체는 Steem 커뮤니티이기 때문에, Steem 네트워크에는 소액 결제가 없다. “좋아요” 투표 여부와 상관없이 동일한 금액의 후원금이 투표자가 아닌 커뮤니티에 의해 지급된다.

            \paragraph{}
                경제적 의사 결정을 내릴 때 받는 스트레스는 대부분의 사람들의 참여를 막는 진입 장벽으로 작용한다.

            \begin{quote}
                    \textit{정보가 폭증하는 디지털 시대에서 우리는 인터넷 콘텐츠와 관련하여 매일 다양한 선택들을 직면하게 되며, 그러한 의사 결정들로 인해 불확실성이 나날이 높아져 가고 있다. 소액 결제 지지자들은 간소화된 구현이 소액 결제의 진입 장벽을 낮추고 사용자 경험을 개선할 수 있을 것이라고 주장한다. 하지만, 그러한 주장은 의사결정 프로세스에 대하여 이중 잣대를 적용하는 것에 지나지 않는다 [2]. 거래 1건으로 의사 결정을 정당화할 수 없는 것이다. \textbf{사용자들이 아무 망설임 없이 승인하는 유일한 거래들은 비용이 전혀 발생하지 않는 거래들뿐}이기 때문에, 금액과 상관없이 모든 소액 거래들은 심리적 비용을 발생시킨다. 또한, 심리적 거래 비용이 특정 한계치까지 상승하며 소액 결제의 장점이 전혀 부각되지 않는다. 예를 들어, 오늘 신문이 1달러라고 생각하는 것은 쉬우나, 각 기사 혹은 단어의 가치가 얼마인지를 추정하는 것은 매우 어려운 일이다. 모든 인터넷 콘텐츠들의 결제가 소액 결제 시스템을 기반으로 세분화된다면 그러한 딜레마는 더욱 심화된다.}\\

                    \textit{- Micropayments: A Viable Business Model
\footnote{Micropayments: A Viable Business Model\newline
\url{http://cs.stanford.edu/people/eroberts/cs181/projects/2010-11/Microp}}}
            \end{quote}

            \paragraph{}
                Steem의 경우, 소액 결제 금액이 콘텐츠 제공자에게 지급되지만, 해당 콘텐츠의 투표에 참여한 사람들에게는 아무런 비용도 발생하지 않는다. 대신, 보상 비용은 신규 토큰들을 통해 지급된다. 누구나 가입이 가능하고 투표에 참여할 수 있으며, 돈을 충분히 벌고 탈퇴할 수 있다 (단, Steem 시스템의 시가 총액에 변동이 없어야 함). 다시 말해서, Steem의 소액 결제 솔루션은 사용자 수정 콘텐츠를 보유한 대중적인 웹사이트들과 유사한 사용자 경험을 제공한다.

            \paragraph{}
                게다가, Steem은 지불 대상자들을 알아내는 사람들에게 그에 상응하는 대가를 지불하고 있다! 이는 매우 혁신적인 발상이라 할 수 있다.

        \subsection{연결 링크에 내포된 가치}

            \paragraph{}
                연결 링크가 완전히 없어질 경우, 인터넷은 상당 부분의 가치를 잃게 될 것이다. Google이 1,600만 검색 결과들 중에서 최상의 애플파이 조리법을 제시할 수 있는 것은 웹 페이지들 간의 상관관계 덕분이다. 연결 링크 없이는 Google이 취합할 수 있는 유일한 정보는 단어 사용 빈도뿐이다.

            \paragraph{}
                연결 링크들은 다양한 형태가 있으며, 지속적으로 진화해왔다. 사용자가 소셜 네트워크 상에서 콘텐츠에 투표를 할 때마다, 해당 사용자와 콘텐츠 간의 연결 링크가 추가된다. 해당 콘텐츠를 통해 소비자와 생산자를 연결한다. 연결 링크들이 많을수록 정보의 가치가 상승한다. 부가가치를 창출하는 것은 상대적/의도적인 정보의 연결성이다.

            \paragraph{}
                연결 링크들이 정량/정성적으로 개선될수록 소셜 네트워크는 콘텐츠 가치를 최대화할 수 있다. 콘텐츠 맞춤화는 많은 비용이 소요되고 시간도 오래 걸리며, 연결 링크 없이는 컴퓨터 성능이 보장될 수 없다. Steem은 신규 콘텐츠의 연결 링크를 최초로 발견한 사용자들을 보상한다.

            \paragraph{}
                큐레이션에 인센티브를 적용함으로써, Steem 네트워크는 자동화 알고리즘을 활용하여 콘텐츠 홍수 속에서 가장 유용한 정보만을 추출할 수 있다.

        \subsection{암호화폐 도입 관련 문제에 대한 해결책}

            \paragraph{}
                누구나 쉽게 암호화폐\footnote{Dailydot,
Jon Southurt, April 2015\newline
\url{http://www.dailydot.com/opinion/bitcoin-cryptocurrency-adoption-hard}}를 사용하기는 어렵다. Bitcoin을 사용해 본 경험자라면 누구나 서비스 가입 후 신용카드 혹은 계좌 이체를 통해 계정을 충전시켜야 한다는 것을 알게 된다. Facebook이 유료 서비스였다면 지금의 성공이 있었을까?

            \paragraph{}
                Steem은 단순하지만 소중한 기여도를 제공한 모든 사람들을 보상함으로써 그러한 문제를 해결한다. 이를 통해 STEEM 토큰들의 보급률이 확대될 것이다. 암호화폐에는 사용자들이 많을수록 확대되는 네트워크 효과가 있기 때문에 그러한 보급률 확대는 더욱 유용하다. 극단적인 예로, 만약 Satoshi가 Bitcoin의 100\%를 보유하고 있었더라면 Bitcoin은 아무런 가치가 없었을 것이다.

        \subsection{암호화폐 유동화 문제에 대한 해결책}

            \paragraph{}
                사용이 어렵거나 매매가 불가능한 화폐는 거의 가치가 없다. \$1.00 가치의 Bitcoin을 우연히 발견한 사람은 해당 코인을 Bitcoin에 되팔려면 \$1.00 이상의 비용이 발생한다는 사실을 알게 될 것이다. 계정 개설, KYC 검증, 수수료 지급 등에 소요되는 비용이 \$1.00 이상이다. 소액 암호화폐는 줍는 것조차 귀찮은 잔돈과 같다.

            \paragraph{}
                기업들은 암호화폐를 신속히 유형 자산/서비스로 전환할 방법을 사용자들에게 제시한다. 기업들은 미국 달러 환율에 고정된 화폐를 필요로 한다. 변동성이 높은 화폐를 도입하려면 상당한 회계 관련 비용이 발생하게 된다.

            \paragraph{}
                매출만 증대된다면 기업들은 어떠한 화폐도 수용할 것이다. SBD와 같은 안정적인 화폐를 보유한 사용자 기반을 광범위하게 보유할 수 있다면 기업들의 진입 장벽이 대폭 낮춰질 것이다. 기업들의 존재는 거래소 없이도 출구 전략의 실행을 가능토록 만든다. 소액 암호화폐를 유동화할 수 있는 또 다른 방법은 후원금을 통해 Steem 플랫폼에 참여하는 것이다. 이는 웨이터에게 팁을 주는 것과 마찬가지다. 소액 후원금이라도 사람들이 많아지면 금액이 커질 수 있다. 손님과 웨이터 모두가 이익을 보는 것이다.

        \subsection{검열}

            \paragraph{}
                Steem은 전 세계에서 엄선된 마이너들에 의해 운영되는 분산 네트워크이다. 모든 사용자 활동들은 블록체인 상에서 공개되므로 공개 검증이 가능하다. 따라서, 누구도 STEEM 투자자들이 소중히 여기는 콘텐츠를 검열할 수 없게 된다.

            \paragraph{}
                steemit.com과 같은 독립적인 웹사이트들이 특정 콘텐츠를 검열할 수도 있지만, 블록체인 상에 게시된 콘텐츠는 검열이 불가능하다.

            \paragraph{}
                언론의 자유는 모든 자유의 근간을 이루며, 언론의 자유가 침해되면 건설적인 합의와 토론이 저하된다. 자유로운 토론 없이는, 투표자 간의 정보 공유가 불공평해지며 그러한 불공평성이 사회를 위협할 수도 있다. 검열은 정보 공개를 제한하여 투표권을 도용하는 수단이다. Steem은 언론의 자유를 지향한다.

        \subsection{검색 엔진 최적화를 통한 유기적 성장}

            \paragraph{}
                대부분의 암호화폐들은 네트워크 사용 빈도가 낮은 사람들에게는 별다른 가치가 없다. 그와 대조적으로, Steem은 다양한 콘텐츠를 생성하여 사용자들 간의 공유를 장려한다. 이러한 콘텐츠들은 많은 검색 엔진들의 검색 결과에 표시되기 때문에 수동적인 사용자들에게도 부가가치를 창출한다. 이러한 검색 트래픽은 Steem 네트워크에 유기적인 광고 효과를 낳으며 네트워크 효과를 증대시킨다.

        \subsection{블록체인 기반 귀속성으로의 회귀}

            \paragraph{}
                인터넷은 전세계에 정보를 가장 손쉽게 배포할 수 있는 매체이다. 동시에, 적절한 귀속성을 원하는 콘텐츠 제공자들에게는 가장 위협적인 매체이기도 하다. 현재의 소셜 미디어 플랫폼 상에서, 귀속성은 하룻밤 사이에 잃을 수 있는 그 무엇이다. 즉, 누구나 원 제작자의 동의 없이 게시된 동영상 혹은 영상을 복제하여 남들과 공유할 수 있다.

            \paragraph{}
                블록체인 기반 소셜 미디어의 경우, 원 제작자 혹은 저자가 공공 거래장부 및 타임 스탬프에 기록되기 때문에 콘텐츠 저작권이 보장된다. 원 제작자의 동의 없이 콘텐츠가 무단 사용된 경우, 블록체인 기반 거래장부를 통해 저작권을 보장받을 수 있다. 미래에는, 블록체인 기반 귀속성의 진위가 정부에 의해 공인되고, 콘텐츠 제공자들의 권리는 더욱 증진될 것이다.

            \paragraph{}
                블록체인 유형과 상관없이 타임 스탬프 서비스의 구현이 가능하고 Bitcoin 네트워크에도 그러한 서비스를 구현할 수 있지만, 콘텐츠 제공자들은 "1급 시민들"이기 때문에 여전히 Steem에 경쟁 우위가 있다고 할 수 있다. Steem 블록체인은 콘텐츠 기반을 토대로 구축되므로, 콘텐츠 제공자들이 블록체인을 활용하여 효과적으로 콘텐츠 저작권을 확보할 수 있다.

        \subsection{광고를 블록체인 기반 콘텐츠 보상으로 대체하는 방법}

            \paragraph{}
                대부분의 콘텐츠 화폐화 모델들의 경우, 콘텐츠 제공자들은 다양한 광고 기법들을 활용한다. 많은 원 제작자들이 광고로 인해 본인 작품의 가치가 하락한다고 생각하지만, 어느 정도의 금전적 보상도 동시에 기대한다. 광고는 양날의 칼과 같다: 광고를 통해, 원 제작자는 손쉽게 돈을 벌 수 있다. 광고 없이는, 수익 창출은 어렵겠지만 콘텐츠 자체의 가치는 빛날 수 있다.

            \paragraph{}
                Steem 연계 소셜 미디어에 본인 작품을 게시하는 원 제작자들은 사용자들의 "좋아요" 투표 만으로도 돈을 벌 수 있다. 블록체인 기반 배당금 지급은 완전히 디지털 방식으로 이뤄지기 때문에, 중간 상인이 전무하다. 따라서, 블록체인 기반 콘텐츠 보상에 의한 수익 창출이 가속화되어 광고 없이도 진입장벽이 낮아질 것이다.

    \section{결론}

        \paragraph{}
            Steem은 암호화폐 및 소셜 미디어 분야의 다양한 도전과제들을 극복할 목적으로 시행된 일종의 시범 사업이다. Steem은 기존의 소셜 미디어 서비스와 차별화된 수익 창출 기회를 콘텐츠 제공자들과 인터넷 독자들에게 제공한다. Steem 내에서, 사용자들은 기여도에 따라 인터넷 상에서 수익을 창출할 수 있다. 이러한 보상은 Steem의 유동성 덕분에 달러 가치를 갖게 되며, Steem 보유자들은 지속 가능한 수익성을 확보하게 될 것이다.

\end{document}
